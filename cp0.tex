\chapter{Hamiltonian Systems on Linear Symplectic Spaces}

\section{Introduction}

\qquad Newton's second law for a particle moving in Euclidean three-dimensional space $\mathbb R^3$, under the influence of a \emph{potential energy} $V(q)$, is
$$
F=ma,
$$
where $q\in \mathbb R^3$, $F(q)=-\nabla V(q)$ is the \emph{force}, $m$ is the mass of the particle, and
$$
a=\frac{d^2q}{dt^2}=\ddot q
$$
is the acceleration (assuming we start in a postulated privileged coordinate frame called an \emph{inertial frame}. The potential energy $V$ is introduced through the notion of work and the assumption that the force field is conservative. The introduction of the \emph{kinetic energy}
$$
K=\frac{1}2 m \left\Vert \dot q\right\Vert^2 
$$
is through the \emph{power}, or \emph{rate of work equation}:
$$
\frac{dK}{dt}=m\langle \dot q, \ddot q\rangle=\langle \dot q, F\rangle,
$$
where $\langle \cdot, \cdot \rangle$ denotes the innter product on $\mathbb R^3$.\\

The \emph{Lagrangian} associated to this system is defined by
$$
L(q^i, \dot q^i)=\frac{m}2 \Vert \dot q\Vert^2-V(q)
$$
and one can check that the Newton's second law is equivalent to the \emph{Euler-Lagrange equations}:
$$
\frac{d}{dt} \frac{\partial L}{\partial \dot q^i}-\frac{\partial L}{\partial q^i}=0,
$$
In fact, it follows from the definition of the Lagrangian that
$$
\frac{d}{dt} \frac{\partial L}{\partial \dot q^i}-\frac{\partial L}{\partial q^i}=m\ddot q^i-\frac{\partial V}{\partial q^i}=(ma+\nabla V)_i.
$$
The Euler-Lagrange equation is a second-order differential equation in $q$. Those equations are worthy of independent study for a general $L$, they are the equations for stationary values of the \emph{action integral}:
$$
\delta \int_{t_1}^{t_2}L(q^i, \dot q^i)dt=0,
$$
as will be detailed later. These \emph{varitational principles} play a fundamental role throughout mechanics (both in particle mechanics and field theory).\\

A simple computation shows that $dE/dt=0$, where $E$ is the \emph{total energy}
$$
E=\frac{1}2m\Vert \dot q\Vert^2+V(q).
$$
Lagrange and Hamilton observed that it is convenient to introduce the momentum $p_i=m\dot q^i$ and rewrite $E$ as a function of $p_i$ and $q^i$ by letting
$$
H(p,q)=\frac{\Vert p\Vert^2}{2m}+V(q).
$$
Newton's second law is equivalent to \emph{Hamilton's canonical equations}
$$
\dot q^i=\frac{\partial H}{\partial p_i},\quad \dot p_i=-\frac{\partial H}{\partial q^i},
$$
which is a first-order differential equations system in $(p,q)$-space, of \emph{phase space}. In fact,
$$
\dot q^i-\frac{\partial H}{\partial p_i}=\dot q^i-\frac{p_i}m=\dot q^i-\frac{m\dot q^i}m,\qquad \dot p_i+\frac{\partial H}{\partial q^i}=\frac{d}{dt}(m\dot q^i)+\frac{\partial V}{\partial q^i}=m\ddot q^{\thinspace i}+\frac{\partial V}{\partial q^i}.
$$
The first equation is equivalent to the definition of momentum and the second one to the Newton's second law. For a deeper understanding of Hamilton's equations we recall some matrix notation. \\

Let $E$ be a real linear space and $E^\ast$ its dual space. Let $e_1,....,e_n$ be a basis of $E$ with the associated dual basis for $E^\ast$ denoted by $e^1,...,e^n$. That is, $e^i$ is defined by the equations
$$
\langle e^i, e_j\rangle=e^i(e_j)=\delta_{ij},
$$
where $\delta_{ij}$ is the Kronocker delta. Throughout these notes we will use Einstein summation convention. That is, when an index variable appears twice in a single term, it implies summation over all the values of the index. In other words, if there is no risk of confusion, summation symbol ($\Sigma$) will be omitted. For example, a vector $v\in E$ can be written as $v=v^ie_i$ and a covector $\alpha\in E^\ast$ as $\alpha=\alpha_ie^i$, where $v^i$ and $\alpha_i$ are the\emph{components} of $v$ and $\alpha$, respectively.\\

Let $E$ and $F$ be linear spaces. If $A:E\to F$ is a linear transformation, its \emph{matrix} relative to bases $e_1,...,e_n$ of $E$ and $f_1,...,f_m$ of $F$ is denoted by $A_i^j$, and is defined by
$$
A(e_i)=A_i^j f_j.
$$
Thus, the columns of the matrix $A$ are $A(e_1),...,A(e_n)$. If $B:E\times F\to \mathbb R$ is a bilinear form, its matrix $B_{ij}$ is defined by $B_{ij}=B(e_i, f_j)$. Define the \emph{associated} linear map $B^\flat:E\to F^\ast$ by
$$
B^\flat(v)(w)=B(v,w)
$$
for all $v\in E$ and $w\in F$, and observe that $B^\flat(e_i)=B_{ij}f^j$. Since $B^\flat(e_i)$ is the $i$-th column of the matrix representing the linear map $B^\flat$, it follows that the matrix of $B^\flat$ in the bases $e_1,...,e_n$ and $f^1,...,f^m$ is
$$
[B^\flat]_{ij}=B_{ji},
$$
that is, it is the transpose of the matrix $B_{ij}$.\\

Let $Z$ denote the linear space of pairs $(p,q)$ and write $z=(p,q)$. Let the coordinates $q^i$, $p_i$ be collectively denoted by $z^I$, with $I=1,2,...,2n$. One reason for the notation $z$ is that if one thinks of $z$ as a ``complex variable'' $z=q+ip$, then Hamilton's equations can be written in a simpler way. Let
$$
\frac{\partial}{\partial \overline z}=\frac{1}2\left(\frac{\partial }{\partial q}-i\frac{\partial}{\partial p}\right).
$$
Then, it follows from the previous notation that
$$
2\frac{\partial H}{\partial \overline z}=\frac{\partial H}{\partial q}-i\frac{\partial H}{\partial p}=-\dot p+i\dot q=i(\dot q+i\dot p)=iz.
$$
Then, the complex version of Hamilton's equations can be simply written as $\dot z=-2i \partial H/\partial \overline z$.\\

Note that the linear space $Z$ inherites an inner product from $\mathbb R^{2n}$. Recall that, if $f:Z\to \mathbb R$ is a differentiable function, then $df_z: T_zZ\to \mathbb R$ is the linear map defined by
$$
df_z(v)=\frac{d}{dt}f(z+tv)\bigg|_{t=0},
$$
where the tangent vector $v\in T_zZ$ can be identified as an element of the linear space $Z$. The gradient of $f$ at $z$, denoted by $\nabla f(z)$, is the unique tangent vector in $T_zZ$ satisfying
$$
df_zv=\langle \nabla f(z), v\rangle.
$$
In other words, $\nabla f(z)$ is the \emph{Riesz representation} of $df_z$ under the inner product. In local coordinates, 
$$
df_z=\frac{\partial f}{\partial q^i}dq^{i}+\frac{\partial f}{\partial p_i}dp_i,
$$
where $z=(q, p)\in Z$. Similarly, the gradient of $f$ at $z$ can be written as
$$
\nabla f(z)=\frac{\partial f}{\partial q^i}\frac{\partial}{\partial q^i}+\frac{\partial f}{\partial p_i}\frac{\partial}{\partial p_i}.
$$
One can identify vectors and covectors via the standard Euclidean inner product. Thus, $df_z$ can be regarded as a row vector with entries $\partial f/\partial z_i$ and $\nabla f(z)$ as a column vector with the same entries.

\section{Symplectic and Poisson structures}

We can view Hamilton's equations as follows. Consider the operation
$$
\nabla H(z)=\left(\frac{\partial H}{\partial q}, \frac{\partial H}{\partial p}\right) \mapsto \left(\frac{\partial H}{\partial p},-\frac{\partial H}{\partial q}\right):=X_H(z),
$$
which forms a vector field $X_H$. This vector field is called the \emph{Hamiltonian vector field}, from the differential of $H$, and is defined as the composition of certain linear map
$$
R:T_zZ\to T_zZ
$$
and the gradient $\nabla H(z)$ of $H$ at $z$. In fact, the matrix of $R$ with respect to the coordinates $z^I$ is
$$
[R]=\begin{pmatrix}0&-1\\1&0\end{pmatrix}=\mathbb J,
$$
where we write $\mathbb J$ for that specific matriz which is sometimes called the \emph{symplectic matrix}. Thus,
$$
X_H(z)=R\cdot \nabla H(z),
$$
or, if the components of $X_H$ are denoted $X^I$, with $I=1,...,2n$,
$$
X^I=R^{IJ} \frac{\partial H}{\partial z^J}.
$$
Alternatively, the equation above can be written as $X_H=\mathbb J\nabla H$, where $\nabla H$ is the gradient of $H$.\\

\noindent \textbf{Note:} Let $\mathbb J\in M_{2n\times 2n}(\mathbb F)$ be the symplectic matrix. If $v\in \mathbb F^n$, then
$$
v^\intercal \mathbb J=(\mathbb J v)^\intercal.
$$

Previously, it was established a natural identification of $1$-forms and tangent vectors. To be more precise, using the natural coordinates, $df_z$ and $\nabla f(z)$ can be regarded as vectors with the same entries. Thus, the map $R$ can be re-interpreted as a map $R: Z^\ast\to Z$ as follows
$$
R(a_idq^i+b_idp_i)=b_idq^i-a_idp_i,
$$
where $a_i, b_i\in \mathbb R$ and the covectors $dp^i$ and $dp_i$ represent the dual basis associated with the coordinates $(q,p)$. 
Let $B(\alpha, \beta)=\langle \alpha, R(\beta)\rangle$ be the bilinear form associated to $R$, where $\langle \cdot,\cdot \rangle$ denotes the canonical pairing between $Z^\ast$ and $Z$. One calls either the bilinear form $B$ or its associated bilinear form $R$, the \emph{Poisson structure}. The classical \emph{Poisson bracket} is defined by
$$
\{F, G\}=B(dF, dG)=(\nabla F)^\intercal \mathbb J\nabla G=\nabla F\cdot \mathbb J\nabla G,
$$
where $F$ and $G$ are smooth functions from the linear space $Z$ to $\mathbb R$. Recall that,
$$
\nabla (f\cdot g)(z)= (Df_z)^\intercal g+(Dg_z)^\intercal f,
$$
where $f,g:\mathbb R^n\to \mathbb R^k$ are differentiable functions. Moreover, $Df_z$ and $Dg_z$ are the $k\times n$ matrices of partial derivatives. Since $D(\nabla F)=\Hess(F)$, the Hessian of the function $F:Z\to \mathbb R$, it follows that, if $F, G:Z\to \mathbb R$ are differentiable functions and $A$ is a constant matrix, then
$$
\nabla (\nabla F\cdot A\nabla G)=\nabla (\nabla F\cdot \nabla (AG))=\Hess(F) \nabla(AG)+\Hess(AG)\nabla F.
$$
Siince Hessian matrix is symmetry, there is no transposes on the left hand side. Now, note that,
\begin{equation*}
\begin{split}
\{F, \{G, H\}\}&=(\nabla F)^\intercal \mathbb J\nabla (\nabla G\cdot  \mathbb J\nabla H)\\
&=(\nabla F)^\intercal \mathbb J(\Hess(G)\mathbb J\nabla H+(\mathbb J\Hess(H))^\intercal\nabla G),\\
&=(\nabla F)^\intercal \mathbb J(\Hess(G)\mathbb J\nabla H+\Hess(H)\mathbb J^\intercal\nabla G),\\
&=(\mathbb J\nabla F)^\intercal \Hess(G)(\mathbb J\nabla H)-(\mathbb J\nabla F)^\intercal \Hess(H)(\mathbb J\nabla G).
\end{split}
\end{equation*}
Similarly, one can easily find the formulas corresponding to rotate $F$, $G$ and $H$:
\begin{equation*}
\begin{split}
\{G, \{H, F\}\}&=(\mathbb J\nabla G)^\intercal \Hess(H)(\mathbb J\nabla F)-(\mathbb J\nabla G)^\intercal \Hess(F)(\mathbb J\nabla H),\\
\{H, \{F, G\}\}&=(\mathbb J\nabla H)^\intercal \Hess(F)(\mathbb J\nabla G)-(\mathbb J\nabla H)^\intercal \Hess(G)(\mathbb J\nabla F).
\end{split}
\end{equation*}
Finally, since each term $(\mathbb J\nabla F)^\intercal \Hess(G)(\mathbb J\nabla H)$ is one-dimensional, it follows that
$$
[(\mathbb J\nabla F)^\intercal \Hess(G)(\mathbb J\nabla H)]^\intercal=(\mathbb J\nabla H)^\intercal \Hess(G)(\mathbb J\nabla F).
$$
Thus, if $F$, $G$ and $H$ are differentiable functions from $Z$ to $\mathbb R$, then
$$
\{F,\{G,H\}\}+\{G,\{H,F\}\}+\{H,\{F,G\}\}=0.
$$
The above property is called the \emph{Poisson identity}.\\

\noindent The \emph{symplectic structure} $\Omega$ is the bilinear form associated to
$$
R^{-1}: Z\to Z^\ast,
$$
that is, $\Omega(v, w)=\langle R^{-1}v,w\rangle$, or equivalently, $\Omega^\flat=R^{-1}$. The matrix of $\Omega$ is $\mathbb J$, in the sense that
$$
\Omega(v,w)=v^\intercal \mathbb Jw.
$$
Summarizing, there are four important maps involving the symplectic matrix:
\begin{itemize}
\item The symplectic form $\Omega: Z\times Z\to \mathbb R$, with matrix $\mathbb J$.
\item The associated linear map $\Omega^\flat: Z\to Z^\ast$, with matrix $\mathbb J^\intercal$.
\item The inverse map $\Omega^\sharp=(\Omega^\flat)^{-1}=R: Z^\ast\to Z$, with matrix $\mathbb J$.
\item The Poisson form, $B:Z^\ast\times Z^\ast\to \mathbb R$, with matrix $\mathbb J$.
\end{itemize}
Thus, Hamilton's equations may be written as follows:
$$
\dot z=X_H(z)=\Omega^\sharp dH(z),
$$
under the identification between $dH(z)$ and $\nabla H(z)$. By multiplying both sides by $\Omega^\flat$, we get
$$
\Omega^\flat X_H(z)=dH(z).
$$
Thus, in terms of the symplectic form, one can write
$$
\Omega(X_H(z), v)=dH(z)\cdot v
$$
for all $z, v\in Z$. Problems such as rigid body dynamics, quantum mechanics as Hamiltonian system, and the motion of a particle in a rotating reference frace motivate the need to generalize these concepts.

































